% Define the page
\documentclass[12pt] {article}
\usepackage[dvips]{graphicx}
\usepackage{epsfig}
\usepackage{amsmath, amsfonts, amsthm}
\setlength{\oddsidemargin}{.2in}
\setlength{\topmargin}{0in}
\setlength{\textwidth}{6.0in}
\renewcommand{\baselinestretch}{1.0}
\newcommand{\field}[1]{\mathbb{#1}}
\newcommand{\R}{\field{R}}
\newcommand{\Z}{\field{Z}}
\newcommand{\N}{\field{N}}
\newcommand{\bt}[1]{\mbox{\boldmath $#1$}}
\newcommand{\sR}{\hbox{I\kern-.1667em\hbox{R}}}
\newtheorem{theorem}{Theorem}[section]
\newtheorem{lemma}{Lemma}[section]
\newtheorem{definition}{Definition}[section]
\newtheorem{remark}{Remark}[section]
\newsavebox{\savepar}
\newenvironment{boxit}{\begin{lrbox}{\savepar}
\begin{minipage}[b]{5.5in}}
{\end{minipage}\end{lrbox}\fbox{\usebox{\savepar}}}
\usepackage{harvard}
\begin{document}
\begin{center}
{\bf User Manual for Reduced Piecewise Exponential Survival Software}
\end{center}
\vspace{.2in}
\begin{center}
\large{July 2011}
\end{center}

\section{General Information}
This reduced piecewise exponential survival software implements
the likelihood ratio test procedure in Han, Schell, and Kim (2009).
Inputs to the program can be either times when events/censoring
occur or the vectors of total time on test and the number of
events. Outputs of the programs are times of events and
the corresponding p-values. The order of times and p-values
is determined by a backward elimination procedure.
Details about the model and implementation are given in
Han, Schell, and Kim (2009). This program can run in
R version $12$ and above.


\section{Inputs and Outputs}
\subsection{Inputs}
This software has one driver files {\tt MPEXEv1}
Inputs to {\tt MPEXEv1} include
\begin{itemize}
\item {\tt `EventTime' = time}, a vector having the times of the
events occurred and of censoring. Note that 'EventTime' is a
required input.
\item {\tt `Censor' = censor}, a vector with {\tt 0} or {\tt 1}
indicating whether an observation is an event or censored. We
use {\tt 0} to denote censoring and {\tt 1} to denote event. The length
of {\tt time} and {\tt censor} are identical. Note that 'Censor'
is a required input.
\item {\tt `CutTimes' = cuttimes}, a vector of the potential
times where the piecewise exponential model is divided. Note that
{\tt `CutTimes'} is an optional input. By default,
{\tt cuttimes} is a vector of times when the events occur.
\item {\tt `Monotone' = monotone}, an indicator with values
and meanings
\begin{itemize}
\item {\tt 0}: no monotonic assumption;
\item {\tt 1}: assuming that the hazard is decreasing over time;
\item {\tt 2}: assuming that the hazard is increasing over time;
\item {\tt 3}: assuming that the hazard is monotonic;
\item {\tt 4}: assuming that the hazard is increasing and then decreasing;
\item {\tt 5}: assuming that the hazard is decreasing and then decreasing;
\item {\tt 6}: assuming that the hazard is increasing and then decreasing with the peak removed first;
\item {\tt 7}: assuming that the hazard is decreasing and then increasing with the peak removed first;
\end{itemize}
Note that {\tt `Monotone'} is an optional input. The default value of {\tt `Monotone'} is $0.$
\end{itemize}
%~~~~~~~~~~~~~~~~~~~~~~~~~~~~~~~~~~~~~~~~~~~~~~~~~~~~~~~~~~~~~~~~~~~
\subsection{Outputs}
Suppose the output is a structure {\tt pexeout}. The two elements are
\begin{itemize}
\item {\tt pexeout.times}: the times where the piecewise exponential model
are divided.
\item {\tt pexeout.pvalues}: p-values corresponding to {\tt pexeout.times}.
Note that {\tt pexeout.times} and {\tt pexeout.times} are in the order of
the backward selection. Each iteration will select a time point having the
largest p-value.
\end{itemize}
\newpage

% ~~~~~~~~~~~~~~~~~~~~~~~~~~~~~~~~~~~~~~~~~~~~~~~~~~~~~~~~~~~~~~~~~~~
\section{Examples}
We demonstrate using {\tt MPEXEv1} in two examples below.
\begin{description}
\item[Example 1. A lung cancer clinical trial at the Moffitt cancer center]
The inputs in this example include times when events/censoring occur and
indicators for censoring, i.e.,

\begin{boxit} {\tt
\noindent
\begin{verbatim}
[eventtime censor] =
1.7211 1.0000
8.0557 1.0000
... ...
0.3827 1.0000
0.2753 0;
\end{verbatim}}
\end{boxit} \\

The command to run the program with hazard decreasing over time assumption is

\begin{boxit} {\tt
\noindent
\begin{verbatim}
trend=1;
pexeout11 = RPEXEv1(EventTime="EventTime",eventtime,
Censor="Censor",censor,Cutime=NA,cuttime=NA,Trend=``Trend",trend);
\end{verbatim}}
\end{boxit} \\

The result is

\begin{boxit} {\tt
\noindent
\begin{verbatim}
pexeout11$times = ...
0.3827
2.7019
22.2418
2.2839
9.0557;
\end{verbatim}}
\end{boxit} \\

\begin{boxit} {\tt
\noindent
\begin{verbatim}
pexeout11$pvalues = ...
0.9839
0.7966
0.3711
0.1746
0.00071;
\end{verbatim}}
\end{boxit} \\

If we assume the hazard is increasing and then decreasing;,
the code and result are listed next.

\begin{boxit} {\tt
\noindent
\begin{verbatim}
trend=4;
pexeout12 = ...
RPEXEv1(EventTime="EventTime",eventtime,Censor="Censor",censor,
Cutime=NA,cuttime=NA,Trend=``Trend",trend);
[pexeout12$times pexeout12$pvalues] =
0.2753 0.9152
1.8925 0.8240
2.7019 0.7966
2.1745 0.7484
22.2418 0.3711
1.7211 0.1946
2.2839 0.1746
8.0557 0.0007
\end{verbatim}}
\end{boxit} \\

\end{description}

% ~~~~~~~~~~~~~~~~~~~~~~~~~~~~~~~~~~~~~~~~~~~~~~~~~~~~~~~~~~~~~~~~~~~
\section{Job Files}
Besides one driver file discussed above ({\tt RPEXEv1},
the package include the files below.
\begin{enumerate}
\item {\tt bisec} runs the bisection algorithm to return a value satisfying
a certain equality. {\tt bisec} is called by {\tt exact$\_$pvalue}.
\item {\tt exact$\_$pvalue} computes the exact p-value of the test statistic
in the likelihood ratio test. {\tt exact$\_$pvalue} is called by both
{\tt loopcuts} and {\tt loopcuts$\_$ttot}.
\item {\tt loopcuts} is the main job file called by {\tt RPEXEv1}.
\item {\tt loopcuts$\_$umbrella} is the main job file used by {\tt RPEXEv1}.
\item {\tt loopcuts$\_$ttot} is the main job filed used by {\tt RPEXEv1}.
\item {\tt pava} returns time points (along with the total time on test and
the number of events) where the hazard is decreasing (or increasing). {\tt pava}
is used if the option {\tt 'monotone'} is set to 1 or 2.
\item {\tt totaltest} takes inputs (times and the indicator of event/censoring) and
returns the total time on test and the number of event. {\tt totaltest} is used
by {\tt loopcuts}.
\end{enumerate}

\begin{center}
{\bf Disclaimer}
\end{center}
This software is given without express or implied
warranty that the result is errorfree or reliable.
The authors accept no responsibility or liability
for loss or damage occasioned by its use.

\begin{center}
{\bf Reference}
\end{center}
\noindent
Han, G., Schell, M. J., and Kim, J. (2009) {\em Improved survival modeling using
a piecewise exponential approach}

\end{document}

