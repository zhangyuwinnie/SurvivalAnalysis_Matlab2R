% Define the page
\documentclass[12pt] {article}
\usepackage[dvips]{graphicx}
\usepackage{epsfig}
\usepackage{amsmath, amsfonts, amsthm}
\setlength{\oddsidemargin}{.2in}
\setlength{\topmargin}{0in}
\setlength{\textwidth}{6.0in}
\renewcommand{\baselinestretch}{1.0}
\newcommand{\field}[1]{\mathbb{#1}}
\newcommand{\R}{\field{R}}
\newcommand{\Z}{\field{Z}}
\newcommand{\N}{\field{N}}
\newcommand{\bt}[1]{\mbox{\boldmath $#1$}}
\newcommand{\sR}{\hbox{I\kern-.1667em\hbox{R}}}
\newtheorem{theorem}{Theorem}[section]
\newtheorem{lemma}{Lemma}[section]
\newtheorem{definition}{Definition}[section]
\newtheorem{remark}{Remark}[section]
\newsavebox{\savepar}
\newenvironment{boxit}{\begin{lrbox}{\savepar}
\begin{minipage}[b]{5.5in}}
{\end{minipage}\end{lrbox}\fbox{\usebox{\savepar}}}
\usepackage{harvard}
\begin{document}
\begin{center}
{\bf User Manual for {\Large R}educed {\Large P}iecewise
{\Large EX}ponential {\Large E}stimate (RPEXE) Software}
\end{center}
\vspace{.2in}
\begin{center}
\large{Gang Han and Michael J. Schell} \\
\large{March 2014}
\end{center}
\vspace{3mm}
\begin{small}
\begin{description}
\item [GH email:] ghan@sph.tamhsc.edu; gang.han@yale.edu
\item [MJS email:] Michael.Schell@moffitt.org
\end{description}
\end{small}

\section{General Information}
RPEXE (Reduced Piecewise EXponential Estimate) is a set of MATLAB programs
or R programs for estimating the survival probability with a reduced
piecewise exponential approach proposed in Han et al. (2014). This set
of programs:
\begin{enumerate}
\item Imposes a pre-specified order restriction on the failure rate, which can be decreasing, increasing, monotone, decreasing then increasing, or increasing
    then decreasing;
\item Computes the p-values (Han et al. 2012) at all event times to compare adjacent total time on tests;
\item Reports the largest (the most non-significant) p-value at each iteration.
\end{enumerate}
After imposing the order restriction the number of candidate changepoints can be
reduced significantly. Using the list of p-values and the critical p-value
 ($\alpha^star$ in Han et al. 2014), one can decide the number and location of changepoints.

This user manual introduces the RPEXE program with an example from Han et al. (2014).
Section~\ref{sec2} sketches the model and implementation. Section~\ref{sec3}
describes inputs/outputs and files used by the program.
Section~\ref{sec4} illustrates RPEXE program with the Non-small-cell lung cancer example in Han, et al. (2014).


\section{Model and implementation}
\label{sec2}

\subsection{Survival Estimates Given the Changepoints in the Failure Rate}
We let $[Z]$
denote the distribution of a generic random variable
$Z$ and $z$ denote a realization from $[Z].$
We let ``log'' denote the natural logarithm.
Suppose $X_e\in (0,\infty)$ is a random variable with the
exponential distribution having parameter $\lambda$.
Suppose $X_{pe}\in (0,\infty)$ is a random variable with
a piecewise exponential distribution divided by
change-points $t_{(1)}<t_{(2)}< \cdots <t_{(p)}$
at which the failure rate varies.

The p.d.f. $f(X_e|\lambda)$,
survival function $S(X_e|\lambda)$, and hazard
function $h(X_e|\lambda)$ are
$f(X_e|\lambda) = \exp \left\{ -X_e/\lambda \right\}/\lambda$,
$S(X_e|\lambda) = \exp \left\{ -X_e/\lambda \right\}$, and
$h(X_e|\lambda) = f(X_e|\lambda)/S(X_e|\lambda) = 1/\lambda.$
Further, the mean and median of $X_e$ are
$E(X_e|\lambda)=\lambda$ and $M(X_e|\lambda) = \lambda \textrm{log}(2).$
Define $t_{\left(0\right)}=0\,$and $t_{(p+1)}=\infty$. Let
$\bt{\lambda}=$ $(\lambda_1,\lambda_2,\ldots,\lambda_{p+1})$ denote
the vector of the unknown model parameters, where $1/\lambda_j$ is the
instantaneous failure rate in $(t_{(j-1)},t_j]$ for
$j=1,\ldots,p+1$. Define $E_0 = 1$ and
$$E_j = \prod_{k=1}^j \exp
\left\{ -\frac{t_{(k)} - t_{(k-1)}}{\lambda_k} \right\}
= E_{j-1} \exp
\left\{ -\frac{t_{(j)} - t_{(j-1)}}{\lambda_j} \right\}.$$
Note that $(t_{(j)}, E_j)$ is the point where the
$j^{\text{th}}$ piece of the $S(X_{pe}|\bt{\lambda})$ starts
in the plot of $S(X_{pe}|\bt{\lambda})$ against $X_{pe}.$
So $\left[(X_{pe}-t_{(j-1)})/E_{j-1}\right]\sim
\textrm{Gamma}(1,\lambda_{j})$
for $X_{pe}\in$ $\left[t_{(j-1)},t_{(j)}\right),$
where $\textrm{Gamma}(1,\lambda_j)$ denotes the gamma distribution
with location parameter $1$ and scale parameter $\lambda_j.$
Thus, $S(X_{pe}|\bt{\lambda}),$
$f(X_{pe}|\bt{\lambda}),$ $h(X_{pe}|\bt{\lambda}),$ and the cumulative
hazard function $H(X_{pe}|\bt{\lambda})$ are
\begin{equation}
\label{survY}
\begin{aligned}
S(X_{pe}|\bt{\lambda}) & =  E_{j-1} \exp \left\{
-\frac{X_{pe}-t_{(j-1)}}{\lambda_j} \right\}, \\
f(X_{pe}|\bt{\lambda}) & =  E_{j-1} \frac{1}{\lambda_j}
\exp \left\{ -\frac{X_{pe}-t_{(j-1)}}{\lambda_j} \right\}, \\
h(X_{pe}|\bt{\lambda}) & =  \frac{1}{\lambda_j}, \\
H(X_{pe}|\bt{\lambda}) & =  -\log(E_{j-1}) + \frac{X_{pe}-t_{(j-1)}}{\lambda_j},
\end{aligned}
\end{equation}
for $X_{pe} \in (t_{(j-1)},t_{(j)}]$ and $j=1,\dots,p+1$.
The mean and median survival times are
$E(X_{pe}|\bt{\lambda})=$
$\lambda_1 + \sum_{i=1}^p(\lambda_{i+1}-\lambda_i)E_i$
and
$\mu_{1/2}(X_{pe}|\bt{\lambda})=$
$t_{(m)} + \lambda_{m+1} \log (2 E_m),$ respectively,
where $m$ is in $\{1,\ldots,p\}$
such that $E_m \geq 0.5$ and $E_{m+1} < 0.5$.

Suppose a study involves $N$ patients, where $D$ of
them failed at time points
$t_1 \leq t_2 \leq \cdots \leq t_D,$ and the other
$(N-D)$ subjects were censored at $t_{D+1},\ldots, t_N$.
Let $t_{1^\star} < t_{2^\star} < \ldots < t_{D^\star}$
denote all the distinct values in $\{t_1,\ldots,t_D\}$
where $D^\star \leq D$. We assume that the censoring
mechanism is independent with the distribution of the
failure time. Define $t_0=0$. The total-time-on-test (TTOT)
between $t_A$ and $t_B$ for $t_A < t_B$ is defined to
be the sum of all subjects' times in $\left(t_A,t_B\right]$
for all $t_A,t_B > 0$; i.e.,
$$\textrm{TTOT}\left(t_A,t_B \right) = \sum^N_{j=1} \max \left[ 0,\,
\min (t_j, t_B) - t_A \right].$$
The normalized spacing between $t_A$ and $t_B$ is defined
as TTOT$\left( t_A,t_B \right)/d_{AB}$, where $d_{AB} > 0$
is the number of events occurred in $(t_A,t_B].$

For exponentially distributed data,
the maximum likelihood estimate (MLE) of $\lambda$ is
$\widehat{\lambda}=\sum^N_{i=1}t_i/D$
and the $100 (1-\alpha) \%$
equal-tailed confidence interval of $\lambda$ is
$$\lambda \in \left[ \frac{2D \widehat{\lambda}}
{\chi_{2D,1-\alpha/2}^2}, \,\,
\frac{2 D \widehat{\lambda}}{\chi_{2D,\alpha/2}^2} \right],$$
where $\chi^2_{2D,\alpha/2}$ is the $100\times \alpha/2 \%$
lower quantile of the central chi-square distribution
with $2D$ degree of freedom.
Similarly, the first and second order derivatives of
the likelihood function of the piecewise exponential
distribution $\log L(\bt{\lambda}|t_1,\ldots,t_N)$
lead to the MLE of $\lambda_j,$ i.e.,
$$\widehat{\lambda}_j = \frac{\textrm{TTOT}(t_{(j-1)},
t_{(j)})}{d_{(j)}}$$ and confidence interval of $\lambda_j$ is
$$\lambda_j \in \left[ \frac{2d_{(j)} \widehat{\lambda}_{j}}
{\chi_{2 d_{(j)},1-\alpha/2}^2}, \,\,
\frac{2 d_{(j)} \widehat{\lambda}_{j}}
{\chi_{2d_{(j)},\alpha/2}^2} \right],$$
for all $j = 1,\ldots,p+1.$ The MLEs of $S(X_{pe}|\bt{\lambda})$, $h(X_{pe}|\bt{\lambda})$,
and $M(X_{pe}|\bt{\lambda})$ are $S(X_{pe}|\widehat{\bt{\lambda}})$,
$h(X_{pe}|\widehat{\bt{\lambda}})$, and $M(X_{pe}|\widehat{\bt{\lambda}}),$
respectively.
The confidence intervals of
$\{S(X_{pe}|\bt{\lambda})$, $h(X_{pe}|\bt{\lambda}),$ $M(X_{pe}|\bt{\lambda})\}$
can be computed by plugging in the lower and upper bounds of $\bt{\lambda}.$
Multiple testing adjustments, e.g., Bonferroni approach, can be
implemented to control the level of a confidence interval
when multiple model parameters are involved.


\subsection{Three Components for Determining the Significant Changepoints}
The key to the proposed RPEXE model is to set change-points
$t_{(1)},\ldots,t_{(p)}$ given the data $\{t_1,\ldots,t_N\}$.
Our approach builds on three components.
The first component is a likelihood ratio test.
Let $x_1$ and $x_2$ denote realizations from
Gamma$(n_1,\lambda_1)$ and Gamma$(n_2,\lambda_2)$, respectively.
We suppose that $n_1$ and $n_2$ are {\em known} positive integers
since the numbers of events will be known positive
integers.
The scale parameters $\lambda_1$ and $\lambda_2$ are {\em unknown}.
The null and alternative hypotheses are
$$
\textrm{H}_0:\lambda_1=\lambda_2\, \textrm{ vs. H}_1:\lambda_1 \neq \lambda_2.
$$
Under the null hypothesis, we let $\lambda_1=\lambda_2=\lambda.$
The likelihood ratio test (LRT) statistic can be derived as
$$\phi\left(x_{1},\, x_{2}\right) =
\left( \frac{x_{1}}{x_{1}+x_{2}}\right)^{n_1}
\left(\frac{x_2}{x_{1}+x_{2}}\right)^{n_2}.$$
The level $\alpha$ likelihood ratio test rejects H$_0$ if
$\phi\left(x_{1},\, x_{2}\right) \leq C_{\alpha},$
where $C_\alpha$ is a real value such that
$P\left( \phi(X_1,X_2) \leq C_\alpha \right)=\alpha$ under H$_0$.
In Han et al. (2012), we quantified the p-value
and proved that this exact likelihood ratio test is the
uniformly most powerful unbiased (UMPU) test of
H$_0$ vs. H$_1$. This test is useful for detecting
failure rate change in a single patient group and for
testing the equivalence of the failure rates
of two exponentially distributed groups.

The second component is a backward elimination procedure,
which is used with the LRT to detect significant changes
in the failure rate over time. Specifically, given
$t_{1^\star} < t_{2^\star} < \cdots < t_{D^\star},$
it is possible to model the data with piecewise exponential
distributions having $1$ to $D^\star$ pieces. We let $\{$T$_1,$T$_2,
\ldots,$T$_{D^\star}\}$
denote $D^\star$ TTOTs in $(0,t_{1^\star}),(t_{1^\star},t_{2^\star}),\ldots,
(t_{D^\star -1},t_{D^\star})$ and $\{d_1,d_2,\ldots,d_{D^\star}\}$
denote the corresponding numbers of events. Thus
$D=\sum_{i=1}^{D^\star} d_i.$ The backward elimination (BE)
procedure has four steps:
\begin{description}
\item[Step 1.] Let $\ell=D^\star.$ Compute $\ell-1$ pairs of
TTOTs, $\left\{(\textrm{T}_1,\textrm{T}_2),\ldots,
(\textrm{T}_{\ell-1},\textrm{T}_{\ell})\right\},$ and the
corresponding pairs of events, $\{(d_1,d_2),\ldots,(d_{\ell-1},d_\ell)\}$
\item[Step 2.] Compute the $\ell-1$ p-values using the LRT for the
$\ell-1$ pairs of TTOTs.
\item[Step 3.] If the largest p-value exceeds a critical value
$\alpha^\star$ and $\ell>1,$ add up the corresponding pair
of TTOTs and pair of events (so that there become $\ell-2$
pairs), and let $\ell=\ell-1$.
\item[Step 4.]
Repeat steps 2 and 3 until the largest p-value is smaller than a
critical value $\alpha^\star$ or $\ell=0.$
\end{description}

The third component is an optional order restriction.
Expert knowledge may suggest that the change in failure
rate follows a pattern. For example, the failure rate of NSCLC
patients can decrease over time.
In such situations, integrating appropriate order restrictions
with a statistical model may be necessary. To implement simple
order restriction, we use the pool-adjacent-violators-algorithm
(PAVA). The RPEXE with non-increasing
and non-decreasing failure rates, which we will call isotonic
RPEXE and antitonic RPEXE, are computed by first implementing
the PAVA to the $D^\star$ normalized spacings
$\{\widehat{\lambda}_1,\ldots, \widehat{\lambda}_{D^\star}\},$
and then running the LRT test of H$_0$ vs. H$_1.$ We let $L$
denote the number of level sets after the PAVA operation.
The resulting $L$ estimates $\widehat{\lambda}_1,\ldots, \widehat{\lambda}_L$
will satisfy the assumed order restriction,
and will be used in the BE procedure. The optional order restriction
on the failure rate includes 1) isotonic, 2) antitonic, 3) monotonic,
4) increasing then decreasing (IDFR), and 5) decreasing then increasing (DIFR).
One can also choose to impose no order restriction.
When the partial order restriction is composed of multiple
orderings (e.g., IDFR order where the shift time from an
increasing to decreasing failure rate is not known), we
use the ordering that maximizes the likelihood.

Integrating the three components, we choose the change-points
by first implementing an order restriction on the failure rate
estimates, and then eliminate all insignificant change-points
detected by the likelihood ratio test using the BE procedure.
The elimination procedure stops when the largest p-value of
all possible change-points is less than a critical value
$\alpha^\star,$ which controls the type I error of the
simultaneous inference of all changepoints.


\subsection{Computation of the Critical Value $\alpha^\star$ }

Given a confidence level $\alpha,$ we estimate the critical value
$\alpha^\star$ by $\widehat{\alpha}^\star$ using a 3-stage Monte
Carlo (MC) procedure:
\begin{description}
\item[Stage 1.] Generate an i.i.d. sample $\{x_1,\ldots,x_D\}$
from Exponential$(1)$ (or Gamma$(1,1)$). Let $D^\star$ denote the number of distinct
values in $\{x_1,\ldots,x_D\}.$
\item[Stage 2.] If an order restriction is imposed, using PAVA
to combine the values that violate the restriction to obtain
$L$ distinct level sets.
\item[Stage 3.] Initialize $\ell = D^\star,$ or $\ell = L$ if
there is an order restriction. Repeat the steps~2-3 in the BE
procedure $(\ell-1)$ times. Save the minimum of the p-values
as $p_{\textrm{min}}$.
\end{description}
Repeat the above 3-stage MC procedure $N_s$ times. Let
$\widehat{\alpha}^\star$ denote the $\alpha$th lower
quantile among the $N_s$ values.
The $95\%$ asymptotic lower and upper bounds of $\alpha^\star$
are the $N_s^l$th and $N_s^u$th smallest p-values among all the
$N_s$ $p_{\textrm{min}}$ values, where
$N_s^l = N_s \times \left(\alpha - z^{0.025} \sqrt{\alpha (1-\alpha)/N_s}\right),$
$N_s^u = N_s \times \left(\alpha + z^{0.025} \sqrt{\alpha (1-\alpha)/N_s}\right),$
and $z^{0.025}$ is the upper $0.025$ quantile of the
standard normal distribution.

Following the aforementioned algorithm, we have numerically
estimated some critical values for practical use (Han et al. 2014).
The log of $\widehat{\alpha}^\star$ values were
found to be linearly related with the log of the event number
$D$ with $R^2>0.9$. Thus we fit two regression lines
of $\log (\widehat{\alpha}^\star)$ on $\log(D)$ for
$\alpha = 0.05$ and $\alpha = 0.2,$ respectively, which can be
used as simple formulas to generate $\widehat{\alpha}^\star$
given $\alpha$ and $D\in[20, 800].$
Table~\ref{tabest2}
shows the estimated regression coefficients
$(\widehat{\beta}_0, \widehat{\beta}_1)$
in regression lines of the form
$$\log (\widehat{\alpha}^\star) = \widehat{\beta}_0 +
\widehat{\beta}_1 \times log(D)$$ for isotonic RPEXE,
monotonic RPEXE, and BE with $\alpha=0.05$ or $\alpha=0.1,$
and $D\in[20,800].$ In practice, $\alpha^\star$ can
simply be estimated as
$$\widehat{\alpha}^\star = \exp(\widehat{\beta}_0) \times
D^{\widehat{\beta}_1},$$
given that the number of event $D$ is between 20 and 800.
\begin{table}[htbp]
    \centering
    %\footnotesize
    \begin{tabular}{|c|c|c|c|c|}
    \hline
        & Isotonic RPEXE & Monotonic RPEXE & Umbrella RPEXE  & BE, no order restriction  \\
    \hline
    $\alpha=0.05$  & $-3.483; -0.380$ & $-4.233; -0.394$
                   & $-1.223; -2.704$ & $-2.356; -1.360$  \\
    \hline
    $\alpha=0.1$   & $-2.670; -0.372$ & $-3.448; -0.385$
                   & $-1.195; -2.024$ & $-1.511; -1.370$   \\
    \hline
    \end{tabular}
\caption{Estimated linear coefficients ``$\widehat{\beta}_0; \widehat{\beta}_1$'' of
the isotonic RPEXE, monotonic RPEXE, umbrella alternative RPEXE,
and BE for $\alpha = 0.05$ and $\alpha = 0.1.$ }
\label{tabest2}
\end{table}

\subsection{Testing Exponentiality}
The RPEXE program can be used to test the exponentiality of an observed set
of survival times because it can detect all significant changepoints in the
failure rate. Without any significant changepoints the failure rate is a constant
and the survival distribution is exponential.

To checking the exponential distribution, one needs to inspect whether any p-value
in the list of (max) p-values from the backward elimination is lower than the
critical value $\alpha^\star.$ If yes, the exponential assumption is violated.
We will incorporate this test in the next version of the RPEXE program.



\section{Job Files and Inputs/Outputs}
\label{sec3}

\subsection{RPEXE in MATLAB}
\label{sec3.1}


This section summarizes the files and the inputs/outputs of the
RPEXE MATLAB program.
Users of the RPEXE R program please go to
\ref{sec3.2}.


\subsubsection{MATLAB Files}

\begin{enumerate}
\item Files for the likelihood ratio test
\begin{itemize}
\item {\tt totaltest.m} takes inputs (times and the indicator of event/censoring) and returns the total time on test and the number of event.

\item {\tt exact$\_$pvalue.m} computes the exact p-value of the test statistic
    in the likelihood ratio test. {\tt exact$\_$pvalue.m} is called by both
    {\tt loopcuts.m}, {\tt loopcuts$\_$ttot.m}, and {\tt loopcuts$\_$umbrella.m}.

\item {\tt bisec.m} runs the bisection algorithm to return a value satisfying
    a certain equality. {\tt bisec.m} is called by {\tt exact$\_$pvalue.m}.
\end{itemize}

\item Files for the backward elimination
\begin{itemize}
\item {\tt loopcuts.m} uses a loop format to search changepoints in the backward elimination. This file is called by {\tt RPEXEv1.m} if the order restriction
    is isotonic, antitonic, monotone, and no restriction.

\item {\tt loopcuts$\_$ttot.m} is the backward elimination job file used by {\tt RPEXEv2.m}. It performs the same job but using the total time on test instead of the event times.

\item {\tt loopcuts$\_$umbrella.m} is the backward elimination job file
    used by {\tt RPEXEv1.m} when the failure rate is increasing-decreasing or
    decreasing-increasing.
\end{itemize}

\item Files for implementing the order restriction
\begin{itemize}
\item {\tt pava.m} uses the event times and censoring status and returns time points (along with the total time on test and the number of events) where the hazard is decreasing (or increasing). {\tt pava.m} is called if any order restriction is specified.

\item {\tt umbrella.m} uses the umbrella alternative to merge certain entries to make sure the sequence of ttot/deaths increasing then decreasing or decreasing then increasing. (Note that the pava function makes the sequence non decreasing.  This function directly uses the pava function.)

\item {\tt gamllik.m} computes the log likelihood from the gamma distribution under
    an order restriction. This function is used to select the trend that maximizes the likelihood.
\end{itemize}

\item Driver files
\begin{itemize}
\item {\tt RPEXEv1.m} is the main job file if the input is a set of event times and the corresponding censoring status. It also requires the order restriction. The output includes the list of changepoints from the backward elimination procedure with the p-values, the final trend information, model information under different trends, and the failure rate peak information under the umbrella alternative.

\item {\tt RPEXEv2.m} is the main job file if the input is a set of total-time-on-tests (instead of event times). The output structure is the same as {\tt RPEXEv1.m}. The current version of the program (ver 1.0) does not allow umbrella alternative. We will add the function in a later version.
\end{itemize}

\end{enumerate}

\subsubsection{Inputs and Outputs}

This package has two driver files {\tt RPEXEv1.m} and
{\tt RPEXEv2.m}, which can take different types of
inputs but return the same outputs.

Inputs to {\tt RPEXEv1.m} include
\begin{itemize}
\item {\tt `EventTime' = time}: a sequence of times at which the events occur. Note that 'EventTime' is a required input.
\item {\tt `Censor' = censor}: a sequence of dichotomous values indicating censored or not (0=censored and 1=not censored).
    We use {\tt 0} to denote censoring and {\tt 1} to denote event. The length of {\tt time} and {\tt censor} are identical. Note that 'Censor' is a required input.
\item {\tt `CutTimes' = cuttimes}, a vector of unique and sorted changepoint candidates. Default is sorted (from small to large) event times. Default == 'EventTime'.
\item {\tt `Trend'}, indicator of the monotonicity assumption
    \begin{itemize}
        \item {\tt 0}: no monotonic assumption;
        \item {\tt 1}: failure rate is decreasing over time;
        \item {\tt 2}: failure rate is increasing over time;
        \item {\tt 3}: monotonic failure rate;
        \item {\tt 4}: failure rate is increasing and then decreasing;
        \item {\tt 5}: failure rate is decreasing and then increasing;
        \item {\tt 6}: failure rate is increasing and then decreasing. The event time corresponding to the highest failure rate will not be considered as a changepoint in order to improve the program stability;
        \item {\tt 7}: failure rate is decreasing and then increasing. The event time corresponding to the highest failure rate will not be considered as a changepoint in order to improve the program stability.
    \end{itemize}
    Default value of 'Trend'== 0 corresponding to no order restriction.
\end{itemize}

Inputs to {\tt RPEXEv2.m} are based on the total time on test and the
number of events:
\begin{itemize}
\item {\tt `Times' = times}, a sequence of times (in ascending order) at which the events occur. For instance, in SEER data sets {\tt times} are the end of each month.
\item {\tt `ttot' = ttot}, a vector containing the total time on test during the time periods indicated by {\tt times}.
\item {\tt `death' = deaths}, the number of events occurred during
    the time periods indicated by {\tt times}.
\item {\tt `CutTimes' = cuttimes}, same as for {\tt RPEXEv1.m},
    {\tt cuttimes} is a vector of unique and sorted changepoint candidates. Default is sorted (from small to large) event times. Default == 'times'.
\item {\tt `Monotone' = monotone}, an input having three levels indicating the monotonic assumption
    \begin{itemize}
        \item {\tt 0}: no monotonic assumption;
        \item {\tt 1}: failure rate is decreasing over time;
        \item {\tt 2}: failure rate is increasing over time.
    \end{itemize}
\end{itemize}

The outputs from both {\tt RPEXEv1.m} and
{\tt RPEXEv2.m} have the same form. Suppose the output is a
structure named ``{\tt pexeout},'' then its elements are
\begin{itemize}
\item {\tt pexeout.times}: changepoint times;
\item {\tt pexeout.pvalues}: p-values corresponding to the times {\tt pexeout.times};
\item {\tt pexeout.trend}:  trend information in text. It is one of the following: `No order restriction,' `Decreasing failure rate,' `Increasing failure rate,' `Monotone failure rate,'
     `Increasing-decreasing failure rate,'  and `Decreasing-increasing failure rate.'
\item {\tt pexeout.struct}:  structure information for multiple order restrictions.
    Each element (i) in the {\tt pexeout.struct} contains the model information under one     order restriction.
    \begin{itemize}
        \item {\tt pexeout.struct(i).time}: sorted time points (from small to large) after the pava algorithm;
        \item {\tt pexeout.struct(i).ttot}: the total-time-on-test for each time point counted as from previous time point (0 for the first time point) to the current time point (the interval includes the current time but not the previous time);
        \item {\tt pexeout.struct(i).death}: the number of events corresponding to the total-time-on-test;
        \item {\tt pexeout.struct(i).loglik}: the log likelihood of based on gamma distribution of the total-time-on-test.
    \end{itemize}
    {\tt pexeout.struct} will be presented only if multiple order restrictions are available (monotonic and umbrella alternative order restrictions).
\item {\tt pexeout.changet}: changepoint time corresponding to lowest or highest failure rate for umbrella alternatives.
    {\tt pexeout.changet} will be presented only under umbrella alternative order restriction.
\end{itemize}
Note that {\tt pexeout.times} and {\tt pexeout.times} are in the order of the backward selection. Each iteration will select a time point having the largest p-value.

\subsection{RPEXE in R}
\label{sec3.2}

This section summarizes the R functions and the inputs/outputs of the
RPEXE R program. Users of the RPEXE MATLAB program please go to
\ref{sec3.1}.


\subsubsection{R Functions}
\begin{enumerate}

\item Functions for the likelihood ratio test
\begin{itemize}
\item {\tt totaltest} takes inputs (times and the indicator of event/censoring) and returns the total time on test and the number of event.
\item {\tt exact$\_$pvalue} computes the exact p-value of the test statistic
    in the likelihood ratio test. {\tt exact$\_$pvalue} is called by both
    {\tt loopcuts}, {\tt loopcuts$\_$ttot}, and {\tt loopcuts$\_$umbrella}.
\item {\tt bisec} runs the bisection algorithm to return a value satisfying
    a certain equality. {\tt bisec} is called by {\tt exact$\_$pvalue}.
\end{itemize}

\item Functions for the backward elimination
\begin{itemize}
\item {\tt loopcuts} uses a loop format to search changepoints in the backward elimination. This file is called by {\tt RPEXEv1} if the order restriction
    is isotonic, antitonic, monotone, and no restriction.
\item {\tt loopcuts$\_$umbrella} is the backward elimination job file
    used by {\tt RPEXEv1} when the failure rate is increasing-decreasing or
    decreasing-increasing.
\end{itemize}

\item Functions for implementing the order restriction
\begin{itemize}
\item {\tt pava} uses the event times and censoring status and returns time points (along with the total time on test and the number of events) where the hazard is decreasing (or increasing). {\tt pava} is called if any order restriction is specified.

\item {\tt umbrella} uses the umbrella alternative to merge certain entries to make sure the sequence of ttot/deaths increasing then decreasing or decreasing then increasing. (Note that the pava function makes the sequence non decreasing.  This function directly uses the pava function.)

\item {\tt gamllik} computes the log likelihood from the gamma distribution under
    an order restriction. This function is used to select the trend that maximizes the likelihood.
\end{itemize}


\item Driver files
\begin{itemize}
\item {\tt RPEXEv1} is the driver function for inputs being a set of event times and the corresponding censoring status. It also requires the order restriction. The output includes the list of changepoints from the backward elimination procedure with the p-values, the final trend information, model information under different trends, and the failure rate peak information under the umbrella alternative.
\end{itemize}

\end{enumerate}


\subsubsection{Inputs and Outputs}

Inputs to the R function {\tt RPEXEv1} has the format
\vspace{-5mm}
\begin{center}
{\tt (EventTime=NA,eventtime,Censor=NA,censor,CutTimes=NA,cuttime=1,Trend=NA,trend=0)},
\end{center}
which is different from the MATLAB format but contains the same input information.

\begin{itemize}
\item {\tt EventTime}: any text string to indicate whether there is event time. If it is left empty (or given value NA) then there is no event time. Default == NA.
\item {\tt eventtime}: a sequence of times at which the events occur. This is a required input.
\item {\tt Censor}: any text string to indicate whether there is censoring status. If it is left empty (or given value NA) then there is no censoring. Default == NA.
\item {\tt censor}: a sequence of dichotomous values indicating censored or not (0=censored and 1=not censored).
    We use {\tt 0} to denote censoring and {\tt 1} to denote event. The length of {\tt time} and {\tt censor} are identical. Note that 'Censor' is a required input.
\item {\tt Cuttimes}: any text string to indicate whether there are candidates of the changepoints. If it is left empty (or given value NA) then there is NO changepoint candidate. Default == NA.
\item {\tt cuttime}, a vector of unique and sorted changepoint candidates. Default is sorted (from small to large) event times. Default == event times.
\item {\tt Trend}, text string indicator about whether there is trend assumption. If left empty (or given value N) then there is no trend specified. Default = NA.
\item {\tt trend}, indicator of the monotonicity assumption
    \begin{itemize}
        \item {\tt 0}: no monotonic assumption;
        \item {\tt 1}: failure rate is decreasing over time;
        \item {\tt 2}: failure rate is increasing over time;
        \item {\tt 3}: monotonic failure rate;
        \item {\tt 4}: failure rate is increasing and then decreasing;
        \item {\tt 5}: failure rate is decreasing and then increasing;
        \item {\tt 6}: failure rate is increasing and then decreasing. The event time corresponding to the highest failure rate will not be considered as a changepoint in order to improve the program stability;
        \item {\tt 7}: failure rate is decreasing and then increasing. The event time corresponding to the highest failure rate will not be considered as a changepoint in order to improve the program stability.
    \end{itemize}
    Default value of 'Trend'== 0 corresponding to no order restriction.
\end{itemize}

The outputs from the R function {\tt RPEXEv1}
have the same format as in MATLAB RPEXE. Suppose the output is a
structure named ``{\tt pexeout},'' then its elements are
\begin{itemize}
\item {\tt pexeout\$times}: changepoint times;
\item {\tt pexeout\$pvalues}: p-values corresponding to the times {\tt pexeout\$times};
\item {\tt pexeout\$trend}:  trend information in text. It is one of the following: `No order restriction,' `Decreasing failure rate,' `Increasing failure rate,' `Monotone failure rate,' `Increasing-decreasing failure rate,'  and `Decreasing-increasing failure rate.'
\item {\tt pexeout\$struct}:  structure information for multiple order restrictions.
    Unlike in MATLAB, {\tt pexeout\$struct} contains the following elements.
    \begin{itemize}
        \item {\tt pexeout\$struct[[1]][[1]]}: combined time points (as a single vector, sorted under each order restriction) after the pava algorithm for all order restrictions;
        \item {\tt pexeout\$struct[[1]][[2]]}: combined total-time-on-tests  (as a single vector, sorted under each order restriction) after the pava algorithm for all order restrictions. For each time point, the total-time-on-test was computed as from the previous time point (0 for the first time point) to the current time point (the interval includes the current time but not the previous time);
        \item {\tt pexeout.struct[[1]][[3]]}: the number of events corresponding to the total-time-on-test;
        \item {\tt pexeout.struct[[1]][[4]]}: the log likelihood values based on gamma distribution of the total-time-on-test under all order restrictions.
    \end{itemize}
    {\tt pexeout\$struct} will be presented only if multiple order restrictions are available (monotonic and umbrella alternative order restrictions).
\item {\tt pexeout\$changet}: changepoint time corresponding to lowest or highest failure rate for umbrella alternatives.
    {\tt pexeout\$changet} will be specified only under umbrella alternative order restrictions.
\end{itemize}
Note that {\tt pexeout\$times} and {\tt pexeout\$times} are in the order of the backward selection where each iteration selects a time point having the largest p-value.


\subsection{Future Work Topics}
We will add the following features in newer version of the program.
\begin{enumerate}
\item Include the umbrella order restriction in the {\tt RPEXEv2.m} file.
\item Incorporating the formula for $\alpha^\star$ in the coding to translate the
naive p-value to the real p-value.
\item Testing the exponentiality of the survival times. Return the p-value and
the estimated power of the test. If one or more significant changepoints are
identified, indicate up to what time the exponential assumption can hold.
\item Given the critical changepoints, report the descriptive statistics.
\end{enumerate}

% ~~~~~~~~~~~~~~~~~~~~~~~~~~~~~~~~~~~~~~~~~~~~~~~~~~~~~~~~~~~~~~~~~~~
\section{A Non-small-cell Lung Cancer Example}
\label{sec4}
We demonstrate using {\tt RPEXEv1.m} in MATLAB and {\tt RPEXEv1} in R
for a non-small-cell lung cancer example (Han et al. 2014).
The hazard rate is believed to decrease over time and the critical value
$\alpha^\star=0.004$. We will demonstrate 1), how to load the data,
2), how to run RPEXE package with the decreasing hazard rate and monotonic
hazard rate assumptions, and 3), how to interpret the results. Kaplan-Meier
curve and fitted models have been visualized in Han et al. (2014).



\subsection{RPEXE in MATLAB}
\subsubsection{Loading Data and Specifying Work Directory}
The data is saved in file ``data.ex1.txt'' in the ``Examples'' directory with 
two columns having the censoring status and event times. MATLAB and R codes 
of this example are saved under the same directory with names ``MATLAB.ex1.m'' 
and ``R.ex1.R'' respectively.

\begin{boxit} {\tt
\noindent
\begin{verbatim}
1.0000 2.9123
1.0000 3.2740
...    ...
1.0000 3.6027
0      31.2192 ;
\end{verbatim}}
\end{boxit} \\

The MATLAB program the program is in
folder ``MATLAB version''. Code for reading the data and adding the program in the working directory is:

\begin{boxit} {\tt
\noindent
\begin{verbatim}
% Include the MATLAB files' directory
addpath('..\MATLAB version');

% Load the data
data_ex1 = textread('data.ex1.txt');
oscensor      = data_ex1(:,1);
ostime        = data_ex1(:,2);
\end{verbatim}}
\end{boxit} \\

\subsubsection{Run RPEXE with Decreasing and Monotonic Failure Rate Assumptions}

Code for running RPEXE with decreasing failure rate and monotonic failure rate is

\begin{boxit} {\tt
\noindent
\begin{verbatim}
% Fit RPEXE with decreasing failure rate assumption
pexeout1     = RPEXEv1('EventTime',ostime,'Censor',oscensor,...
'Trend',1);

% Fit RPEXE with monotonic failure rate assumption
pexeout2     = RPEXEv1('EventTime',ostime,'Censor',oscensor,...
'Trend',3);
\end{verbatim}}
\end{boxit} \\

\newpage
\subsubsection{Results}
Results under the decreasing failure rate assumption:

\begin{boxit} {\tt
\noindent
\begin{verbatim}
pexeout1
[pexeout1.times pexeout1.pvalues]
% pexeout1 =
%       trend: 'Decreasing failure rate'
%       times: [9x1 double]
%     pvalues: [9x1 double]
% ans =
%     1.8932    0.9780
%    23.4603    0.9608
%    47.8548    0.7539
%    12.7096    0.7045
%    24.7096    0.6876
%     0.0849    0.6013
%     2.3863    0.3060
%    18.5945    0.0852
%    28.0301    0.0000
\end{verbatim}}
\end{boxit} \\
We can see that the time sequence of backward elimination and corresponding p-values. The last time 28.0301 (month) has the naive p-value < 0.004 so that it is a significant changepoint.
\newpage

Results under the monotonic failure rate assumption:

\begin{boxit} {\tt
\noindent
\begin{verbatim}
pexeout2
[pexeout2.times pexeout2.pvalues]
% pexeout2 =
%      struct: [1x2 struct]
%       trend: 'Monotone failure rate'
%       times: [9x1 double]
%     pvalues: [9x1 double]
% ans =
%     1.8932    0.9783
%    23.4603    0.9608
%    47.8548    0.7539
%    12.7096    0.7045
%    24.7096    0.6877
%     0.0849    0.6010
%     2.3863    0.3060
%    18.5945    0.0852
%    28.0301    0.0000
\end{verbatim}}
\end{boxit} \\
We can see that the time sequence in the backward elimination and the
p-values are the same as the decreasing failure. Next we check the structure.

\begin{boxit} {\tt
\noindent
\begin{verbatim}
pexeout2.struct
pexeout2.struct(1)
[pexeout2.struct(1).time pexeout2.struct(1).ttot ...
pexeout2.struct(1).deaths]
pexeout2.struct(2)
[pexeout2.struct(2).time pexeout2.struct(2).ttot ...
pexeout2.struct(2).deaths]
% ans =
% 1x2 struct array with fields:
%     time
%     ttot
%     deaths
%     loglik
% ans =
%       time: [10x1 double]
%       ttot: [10x1 double]
%     deaths: [10x1 double]
%     loglik: -543.9703
% ans =
%    1.0e+03 *
%     0.0001    0.0151    0.0020
%     0.0019    0.3021    0.0250
%     0.0024    0.0733    0.0060
%     0.0127    1.0907    0.0760
%     0.0186    0.3450    0.0220
%     0.0235    0.1992    0.0100
%     0.0247    0.0412    0.0020
%     0.0280    0.0997    0.0040
%     0.0479    0.3699    0.0060
%     0.0511    0.0825    0.0010
% ans =
%       time: [2x1 double]
%       ttot: [2x1 double]
%     deaths: [2x1 double]
%     loglik: -559.4657
% ans =
%    1.0e+03 *
%     0.0006    0.1130    0.0050
%     0.0511    2.5057    0.1490
\end{verbatim}}
\end{boxit} \\
We can see that the structure has two elements corresponding to
decreasing failure and increasing failure. In each element we can see the (sorted) time points after PAVA, the TTOT, the numbers of events, and the log likelihood. Under the decreasing failure assumption we can see 10 time points and 10 TTOTs, and the log likelihood is -543.9703. Under the increasing failure assumption there are only 2 time points left after the PAVA, and the log likelihood value is -559.4657, which is less than that under the decreasing failure.


\subsection{RPEXE in R}
\subsubsection{Loading Data and Specifying Work Directory}
The data is saved in file ``data.ex1.txt'' in the ``Examples'' directory with two columns having the censoring status and event times. The R program is saved in folder ``R version'' with the name ``RcodeREPEXEv1.txt''.

The following code can load the data and point to the source code

\begin{boxit} {\tt
\noindent
\begin{verbatim}
# Set your current work directory that contains this code
setwd("your current work directory")

# include the R functions from the command line
source("../R version/RcodeREPEXEv1.txt")

# load the data
data_ex1 <- read.table("data.ex1.txt", header=F)
oscensor <- data_ex1[,1]
ostime   <- data_ex1[,2]
\end{verbatim}}
\end{boxit} \\

\newpage
\subsubsection{Run RPEXE with Decreasing and Monotonic Failure Rate Assumptions}

Code for running RPEXE with decreasing failure rate and monotonic failure rate is

\begin{boxit} {\tt
\noindent
\begin{verbatim}
# Run RPEXE with decreasing failure rate
pexeout1 <- RPEXEv1(EventTime="EventTime",ostime,
Censor="Censor", oscensor,Trend="Decreasing",trend=1)

# Run RPEXE with monotonic failure rate
pexeout2 <- RPEXEv1(EventTime="EventTime",ostime,
Censor="Censor",oscensor,Trend="monotonic",trend=3)
\end{verbatim}}
\end{boxit} \\


\subsubsection{Results}
Results under the decreasing failure rate assumption:

\begin{boxit} {\tt
\noindent
\begin{verbatim}
pexeout1
#$times
#[1]  1.8932 23.4603 47.8548 12.7096 24.7096  0.0849
 2.3863 18.5945 28.0301
#$pvalues
#[1] 9.782674e-01 9.608166e-01 7.538902e-01 7.045443e-01
 6.876717e-01 6.010137e-01 3.059760e-01
#[8] 8.523530e-02 9.099848e-07
#$trend
#[1] "Decreasing falilure rate"
#$struct
#NULL
#$changet
#NULL
\end{verbatim}}
\end{boxit} \\
We can see that the time sequence of backward elimination and corresponding p-values. The last time 28.0301 (month) has the naive p-value 9.099848e-07 < 0.004 so that it is a significant changepoint.
\newpage

Results under the monotonic failure rate assumption:

\begin{boxit} {\tt
\noindent
\begin{verbatim}
pexeout2
#	$times
#	[1]  1.8932 23.4603 47.8548 12.7096 24.7096
0.0849  2.3863 18.5945 28.0301
#	$pvalues
#	[1] 9.782674e-01 9.608166e-01 7.538902e-01
7.045443e-01 6.876717e-01 6.010137e-01 3.059760e-01
#	[8] 8.523530e-02 9.099848e-07
#	$trend
#	[1] "Monotone failure rate"
\end{verbatim}}
\end{boxit} \\
We can see that the time sequence in the backward elimination and the
p-values are the same as the decreasing failure. Next we check the four elements of the structure.

\begin{boxit} {\tt
\noindent
\begin{verbatim}
#	$struct[[1]][[1]]
#	 [1]  0.0849  1.8932  2.3863 12.7096 18.5945 23.4603
24.7096 28.0301 47.8548 51.0767  0.6438 51.0767
#	$struct[[1]][[2]]
#	 [1]   15.0794  302.0514   73.3403 1090.6864  345.0394
199.2019 41.2268   99.7137  369.8969 82.4547  112.9527 2505.7382
#	$struct[[1]][[3]]
#	 [1]   2  25   6  76  22  10   2   4   6   1   5 149
#	$struct[[1]][[4]]
#	          [,1]      [,2]
#	[1,] -543.9703 -559.4657
#	$changet
#	NULL
\end{verbatim}}
\end{boxit} \\
Unlike in MATLAB, the time points after PAVA ({\tt \$struct[[1]][[1]]}), TTOTs ({\tt \$struct[[1]][[2]]}) and numbers of events ({\tt \$struct[[1]][[3]]}) are three vectors as the combination of results under decreasing failure rate and increasing failre rate. So their lengths are 12=10+2. The two log liklhood values are saved in {\tt \$struct[[1]][[4]]}) being -543.9703 for decreasing fauilure (trend = 1) -559.4657 and increaing failure (trend = 2). Results from the R program are identical to that in the MATLAB program.




\begin{center}
{\bf Disclaimer}
\end{center}
This set of programs is free software:
you can redistribute it and/or modify it
under the terms of the GNU Lesser General Public License as published by
the Free Software Foundation, either version 3 of the License, or any
later version.

This program is distributed in the hope that it will be useful, but
WITHOUT ANY WARRANTY and without even the implied warranty of
MERCHANTABILITY or FITNESS FOR A PARTICULAR PURPOSE.  The author
accepts no responsibility or liability for any loss or damage
occasioned by its use.  See the GNU Lesser General Public License for
more details (http://www.gnu.org/licenses/).



\begin{center}
  {\bf Reference}
\end{center}
\noindent
Han, G., Schell, M., and Kim, J. (2012) ``Comparing Two Exponential Distributions Using the Exact Likelihood Ratio Test,'' \textit{Statistics in Biopharmaceutical Research}, 4(4), 348-356.
\\

\noindent
Han, G., Schell, M., and Kim, J. (2014) ``Improved Survival Modeling in Cancer Research Using a Reduced Piecewise Exponential Approach,'' \textit{Statistics in Medicine}, 33(1), 59-73.

\end{document}

